% -*- latex -*-
% c1.tex
% LabSK 15.03.2022

% Wymagane pola (komentarz):
% Dawid Chmielewski				% Imię Nazwisko
% Praca w linii poleceń		% Temat ćwiczenia


\documentclass[a4paper,11pt]{article}

\usepackage[T1]{polski}
\usepackage[utf8]{inputenc}  			% Kodowanie pliku

\hoffset=-3.0cm                         % Mniejszy lewy margines
\textwidth=18cm                         % szerzej
\evensidemargin=0pt

\voffset=-3cm                           % Mniejszy górny margines
\textheight=27cm                        % szerzej wzdłuż

\setlength{\parindent}{0pt}             % Paragraf od początku linii
\setlength{\parskip}{\medskipamount}    % Odstęp pomiędzy paragrafami
\raggedbottom                           % bez rozciągania strony

% Dodatkowe komendy
\newcommand\BS{\char`\\}                % \BS == back-slash
\newcommand\TY{\raise.17ex\hbox{$\scriptstyle\mathtt{\sim}$}}   % \TY == większa tylda w \tt

\thispagestyle{empty}			        % bez numeracji stron
\usepackage{enumerate}


\title{ Sieci komputerowe - sprawozdanie }
\author{ Dawid Chmielewski, numer indeksu: 311188 }
\date{30 marca 2022}

\begin{document}

\maketitle{Temat ćwiczenia: Konfiguracja środowiska. Diagnostyka adresów IP.}

\section{Ogólny cel ćwiczenia}

Ćwiczenie obejmowało konfigurację środowiska pracy poprzez instalację oraz aktualizację odpowiednich pakietów, tworzenie aliasów oraz połączeń tunelowych SSH oraz diagnostykę adresów IP.

\section{Instalacja i zarządzanie oprogramowaniem w środowisku pracy - Chocolatey}

Ważną częścią konfiguracji środowiska pracy jest instalacja lub aktualizacja niezbędnego oprogramowania. Przydatnym tutaj programem jest menadżer pakietów Chocolatey. Zainstalowana na moim komputerze wersja nie była tą najnowszą, w związku z tym konieczne bylo jej zaktualizowanie.

\begin{verbatim}
    PS C:\Users\Dawid> choco -v
    1.0.0
    PS C:\Users\Dawid> choco upgrade chocolatey
\end{verbatim}

Ostatecznie program został pomyślnie zaaktualizowany do najnowszej dostępnej wersji, czyli 1.0.1:

\begin{verbatim}
    PS C:\Users\Dawid> choco -v
    1.0.1
    PS C:\Users\Dawid>
\end{verbatim}

Chocolatey pozwala na wyświetlanie listy pakietów, dla których istnieją nie zainstalowane nowsze wersje. Użytkownik ma dostęp do tej listy za pomocą polecenia {\tt outdated}:

{\tt
\begin{verbatim}
    PS C:\Users\Dawid> choco outdated
    Chocolatey v1.0.1
    Outdated Packages
    Output is package name | current version | available version | pinned?

    vim|8.2.4632|8.2.4643|false

    Chocolatey has determined 1 package(s) are outdated.
\end{verbatim}
}

Aktualizacja pakietów jest możliwa z poleceniem {\tt upgrade}. Możemy zarówno wskazać wszystkie z nich ({\tt upgrade all}), jak i pojedynczy, wskazując jego nazwę.

{\tt
\begin{verbatim}

    PS C:\Users\Dawid> choco upgrade vim
    Chocolatey v1.0.1
    Upgrading the following packages:
    vim
    By upgrading, you accept licenses for the packages.
    
    (...)

\end{verbatim}
}

\section{Rejestracja w ZeroTier}

Jednym z zadań wykonanych w ramach ćwiczenia było zarejestrowanie się w ZeroTier. Konieczne było w tym przypadku podanie mojego unikalnego ID. 

{\tt
\begin{verbatim}

    volt% zerotier-add ba78ba8de6
    rejestracja stacji ba78ba8de6 w sieci ZET (83048a0632d2b7a6) ...

\end{verbatim}
}

Efektem poprawnej rejestracji było utworzenie pliku {\tt .zet.id}, który przechowywał mój unikalny ID. Plik ten znajduje się w moim katalogu domowym.

{\tt
\begin{verbatim}

    volt% ll -a .zet.id
    -rw-r-----  1 chmield2  stud  11 27 mar 11:12 .zet.id
    volt% cat .zet.id
    ba78ba8de6
    
\end{verbatim}
}

\section{Aliasy i połączenia tunelowe SSH}

Przy częstym używaniu SSH, każdorazowe logowanie się przy używaniu pełnego adresu domeny (tutaj: {\tt volt.zet.pw.edu.pl}) staje się uciążliwe. Rozwiązaniem tego problemu jest stworzenie pliku {\tt config} w folderze {\tt .ssh} Zawartość mojego pliku jest następująca:

\begin{verbatim}
    PS C:\Users\Dawid> cat .\.ssh\config
    Host vol
        Hostname volt.zet.pw.edu.pl
        User chmield2
    Host s*
        ProxyJump vol
        User live
\end{verbatim}

Po stworzeniu tego pliku mogłem logować się na volta za pomocą krótkiego polecenia {\tt ssh vol} oraz łączyć się tunelowo ze stacjami w pracowni (np. polecenie {\tt ssh live@s1}).


\section{Diagnostyka sieci IP}

Najważniejsze komendy stosowane podczas diagnostyki sieci IP to:

\begin{itemize}
    \item ifconfig - dotyczy systemów Linux- wyświetlanie oraz konfiguracja interfejsów sieciowych. 
    Dostępne interfejsy na volcie: bge0, bge1, em0, em1, lo0, lagg0, vlan172, bridge0, ipfw0;
    
    \item ipconfig - polecenie analogiczne do {\tt ifconfig}, jednak stosowane w systemach Windows;
    
    \item Get-NetIPAddress- pozwala na wyświetlenie interfejsów sieciowych. Poniżej prezentuję pierwszy wyświetlony rezultat dla rodziny IPv4.
    
    {\tt
    \begin{verbatim}
    PS C:\Users\Dawid> Get-NetIPAddress -AddressFamily IPv4

    IPAddress         : 172.31.146.82
    InterfaceIndex    : 64
    InterfaceAlias    : ZeroTier One [83048a0632d2b7a6]
    AddressFamily     : IPv4
    Type              : Unicast
    
    (...)
    \end{verbatim}
    }
    
    \item ping - polecenie testujące połączenie między maszyną testującą oraz testowaną; przydatne np. przy testowaniu połączenia internetowego poprzez podanie domeny.
    
    {\tt
    \begin{verbatim}
    PS C:\Users\Dawid> ping wp.pl

    Pinging wp.pl [212.77.98.9] with 32 bytes of data:
    Reply from 212.77.98.9: bytes=32 time=14ms TTL=52
    \end{verbatim}
    }
    
\end{itemize}

\end{document}